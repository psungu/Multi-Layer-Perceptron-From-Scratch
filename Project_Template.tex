\documentclass[12pt]{article}

\usepackage{amsmath,amsthm,amssymb,multirow,paralist}

% Use 1/2-inch margins.
\usepackage[margin=1in]{geometry}
\usepackage{hyperref}
\usepackage{color}
\usepackage{graphicx}

\begin{document}
\begin{flushleft}
Name Surname: \\
Student Number:
\end{flushleft}
\begin{center}
{%\Large 
\textbf{CMPE 597 Sp. Tp. Deep Learning} \\
{\bf Spring 2021 Project I}\\
Due: April 30 by 11.59pm}%\\\linethickness{1mm}\line(1,0){500}
\end{center}



\section*{Answers}
\begin{enumerate}
\item (15 pts) Write the mathematical expressions for forward propagation.
\item (15 pts) Write the mathematical expressions of the gradients that you need to compute in backward propagation.
\item (20 pts) Implement a Network class, \url{Network.py}, where you have the forward, backward propagation, and the activation functions. Use matrix-vector operations.
\item (20 pts) Implement a main function, \url{main.py}, where you load the dataset, shuffle the training data and divide it into mini-batches, write the loop for the epoch and iterations, and evaluate the model on validation set during training. Report the training and validation accuracy.
\item (5 pts) Implement an evaluation function, \url{eval.py}, where you load the learned network parameters and evaluate the model on test data. Report the test accuracy. 
\item After obtaining 16 dimensional embeddings
\begin{enumerate}
\item (10 pts) Create a 2-D plot of the embeddings using t-SNE which maps nearby 16 dimensional embeddings close to each other in the 2-D space. You can use of the shelf t-SNE functions. Implement a \url{tsne.py} file where you load model parameters, return the learned embeddings, and plot t-SNE. Use the words in the \url{vocab.txt} as the labels in the plot. 
\item (5 pts) Look at the plot and find a few
clusters of related words. What do the words in each cluster have in common?
\item (5 pts) Pick the following data points; 'city of new', 'life in the', 'he is the'. Use the model to predict the next word. Does
the model give sensible predictions?
\end{enumerate}
\item (5 pts) Provide a README file where I can find the steps to train, load, and evaluate your model.
\end{enumerate}



\end{document}
